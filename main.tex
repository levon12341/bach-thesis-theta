\documentclass{article}
\usepackage[english]{babel}
\usepackage{amsmath,amssymb, amsthm}

\usepackage{biblatex}
\addbibresource{sample.bib}
\usepackage{csquotes}

\usepackage[margin=1in]{geometry}

\newcommand{\ZZ}{\mathbb{Z}}
\newcommand{\RR}{\mathbb{R}}
\newcommand{\CC}{\mathbb{C}}
\newcommand{\ch}[2]{\left[\begin{array}{c}#1\\ #2 \end{array}\right]}
\newcommand{\de}{\delta}
\newcommand{\ep}{\varepsilon}
\newcommand{\br}[1]{\left(#1\right)}

\usepackage{hyperref}
\usepackage{xcolor}
\hypersetup{colorlinks=true,linkcolor=blue,citecolor=teal,filecolor=magenta,urlcolor=cyan}

\setcounter{MaxMatrixCols}{20}

\theoremstyle{plain}
\newtheorem{theorem}{Theorem}[section]
\newtheorem{lemma}[theorem]{Lemma}
\theoremstyle{definition}
\newtheorem{definition}[theorem]{Definition}

\begin{document}

%%
%% Title page
%%

\begin{center}
{\scshape Federal State Autonomous Educational Institution\\
for the Higher Education\\
National Research University ``Higher School of Economics''\\[1ex]
Faculty of Mathematics\par}

\par\vfill

\textbf{\large Minasian Levon Lermentovich}

\vspace{1.5cm}

{\Large\bfseries
On Certain Lattice Theta Series
\par}

\vspace{1.5cm}

\textbf{\large Bachelor's thesis}

\vspace{1cm}

Field of study: 01.03.01 --- Mathematics,\\[1ex]
Degree programme: bachelor's educational programme ``Mathematics''
\par\vfill
\noindent\parbox[t]{0.48\textwidth}{%
Reviewer:\\[3pt]
Doctor of Sciences,\\
Aleksey Vasilyevich Sleptsov
}\hspace{0.04\textwidth}\parbox[t]{0.48\textwidth}{%
Scientific supervisor:\\[3pt]
Candidate of Sciences,\\
Petr Igorevich Dunin-Barkowski\\[2ex]
%% Uncomment if needed
%Advisor:\\[3pt]
%Doctor of Sciences, professor\\
%Nikolay Nikiforovich Nikolaev
}%
\par\vfill\vfill
Moscow 2024
\end{center}
\thispagestyle{empty}
\pagebreak
%%
%% ===========================================================================
%%

\begin{abstract}
    This work partially establishes the connection between polynomials of Riemann theta constants
    and lattice theta series of 24-dimensional unimodular lattices. 
    Both the polynomials of Riemann theta constants and lattice theta series of 16-dimensional
    unimodular lattices were used as modular forms of weight-8 in the 
    modular form approach in construction of superstring measures.
    It was shown that in case for 16-dimensional unimodular lattices
    there are very elegant relations
    between the two.
    We aimed to generalize the question for 24-dimensional unimodular lattices and were able 
    to partially answer it.
\end{abstract}

\section{Introduction}
In this work we study the relation between 2 different ways of expressing certain modular forms:
via polynomials of Riemann theta constants and via lattice theta series.
The motivation of studying these specific modular forms comes from physics.
The problem of finding NSR superstring measures is a long-standing problem in superstring theory.
One of the approaches is to guess the answer based on known mathematical requirements
it should satisfy.
In case of low genera the problem can be reformulated in the terms of modular forms:
one needs to construct the weight-8 modular forms of certain properties.
We consider 2 different ways of constructing these modular forms:
\begin{itemize}
    \item via polynomials of degree-16 in Riemann theta-constants (so-called Grushevsky ansatz \cite{grushevsky})
    \item  via lattice theta series of 16-dimensional unimodular lattices (so-called OPSMY ansatz \cite{opsmy})
\end{itemize}
It turned out \cite{dunin-barkowski} that there is a very natural connection between the two.
The question arises: what could be said about the analogous constructions for weights other than 8?

In this work we try to address this question. We consider the weight-12 modular forms
constructed similarly to original weight-8 Grushevsky functions
(we refer to these new weight-12 expressions as \emph{weight-12 Grushevsky functions})
and lattice theta series correponding to 24-dimensional unimodular lattices.

We were able to partially solve this problem: we have identified the underlying lattice 
for 3 out of 4 polynomials among weight-12 Grushevsky functions.
One of these answers is rather trivial, the second one easily follows from known results,
and the third one is nontrivial, was not considered in literature before, and is 
the main result of this work.

We start by providing all the necessary definitions for 
modular forms (section \ref{sec:modular-forms}), 
Riemann theta constants (section \ref{sec:riemann-theta-constants}),
and lattice theta series (section \ref{sec:lattice-theta-series}).
Then, in section \ref{sec:main}, we formulate the known result in case of 16-dimensional lattices
and proceed by proving our own results in the case of 24-dimensional lattices.

\section{Modular forms}\label{sec:modular-forms}

We begin with definitions of modular forms, modular group $\Gamma^{(g)}$ and 
its important subgroup $\Gamma^{(g)}(1, 2)$.
Both Grushevsky ansatz and OPSMY ansatz are interesting
exactly because of their modular properties.
\begin{definition}[Modular group]
    Modular group $\Gamma^{(g)}=\text{Sp}(2g, \ZZ)$ is defined as
    \begin{align}
        \Gamma^{(g)}=\text{Sp}(2g, \ZZ) = \{ 
            M \in \ZZ^{2g \times 2g} : 
            M^t \Omega M = \Omega
        \}
    \end{align}
    where $\Omega$ is defined to be
    \begin{align}
    \Omega = 
    \begin{bmatrix}
        0 & I_g \\
        -I_g & 0
    \end{bmatrix}
    \end{align}
\end{definition}

\begin{definition}[Siegel half-space]    
    Siegel half-space $\mathcal{H}^{(g)}$ is defined to be the set of all symmetric $\CC^{g\times g}$
    matrices with positive definite imaginary part.
\end{definition}

\begin{definition}[Action of $\Gamma^{(g)}$]
    Modular group $\Gamma^{(g)}=\text{Sp}(2g, \ZZ)$ acts on Siegel half-space $\mathcal{H}^{(g)}$
    the following way:
    \begin{align}
        \gamma &= \begin{bmatrix}A & B \\ C & D\end{bmatrix} \in \Gamma^{(g)} \\
        \gamma: \tau &\mapsto (A\tau + B)(C\tau + D)^{-1}
    \end{align}
\end{definition}
\begin{definition}[Modular form]
    A holomorphic function $f$ on $\mathcal{H}^{(g)}$ is called a modular form 
    of weight $k$ with respect to subgroup $\Gamma' \le \Gamma^{(g)}$ if
    \begin{align}
        \forall \gamma \in \Gamma' \quad f(\gamma \tau) = \det (C\tau + D)^k f(\tau)
    \end{align}
\end{definition}

\begin{definition}[$\Gamma^{(g)}(1,2)$]
    The subgroup $\Gamma^{(g)}(1,2)$ consists of all the elements 
    \begin{align}
        \gamma &= \begin{bmatrix}A & B \\ C & D\end{bmatrix} \in \Gamma^{(g)}
    \end{align}
    of the modular group $\Gamma^{(g)}$ s.t. all the diagonal elements 
    of $AB^t$ and $CD^t$ are even.
\end{definition}

\section{Riemann theta constants} \label{sec:riemann-theta-constants}
Now we are going to define 
the notion of Riemann theta constants: the basic blocks of Grushevsky functions.
Then we define the set of functions, linear combinations of which form the Grushevsky ansatz for 
superstring measures: weight-8 Grushevsky functions.
At the end of this section we define the generalized construction: 
weight-12 Grushevsky functions.
\begin{definition}[Riemann theta constant with characteristic]
    The Riemann theta constant with characteristic $e=\ch {{\vec{\de}}}{{\vec{\ep}}}$ 
    is the function on Siegel half-space $\mathcal{H}^{(g)}$ defined as
    \begin{align}\label{eq:riemann-theta-const}
        \theta_e(\tau)=
        \theta \ch{{\vec{\de}}}{{\vec{\ep}}} (\tau) =
        \sum_{\vec n \in \mathbb{Z}^g} 
            \exp {
                \Big(
                    \pi \big(\vec n + \frac{{\vec{\de}}}{2}\big)^t \tau \big(\vec n + \frac{{\vec{\de}}}{2}\big) + 
                    \pi i \big(\vec n + \frac{{\vec{\de}}}{2}\big)^t {\vec{\ep}}
                \Big)
            }
    \end{align}
    where ${\vec{\de}}, {\vec{\ep}}$ are binary vectors of length $g$ (i.e. $\in \ZZ_2^g$).
\end{definition}

Although those are called "constants", they are functions of an argument 
$\tau \in \mathcal{H}^{(g)}$. 
Usually theta function is considered to have 2 arguments: $\theta=\theta(z, \theta)$. And
when we substitute $z=0$ we get a function of 1 argument defined in \eqref{eq:riemann-theta-const}.


\begin{definition}[$\Gamma_e$]
    For characteristic $e=\ch{{\vec{\de}}}{{\vec{\ep}}}$ we define the subgroup $\Gamma_e \leq \Gamma^{(g)}$ as 
    \begin{align}
        \Gamma_e = \gamma[e] \Gamma_g(1, 2) \gamma[e]^{-1}
    \end{align}
    where $\gamma[e]$ is an element of $\Gamma^{(g)}$ which maps the Riemann theta constant 
    with zero characteristic to characteristic $e$.
\end{definition}

\begin{theorem}[\cite{mumford-theta}]\label{thm:theta-modularity}
    For every $\gamma \in \Gamma(1, 2)$
        \begin{align}
            \theta\ch{D{\vec{\de}}-C{\vec{\ep}}}{-B{\vec{\de}}+A{\vec{\ep}}}\br{\gamma\tau} = 
            \zeta_{\gamma} \det\br{C\tau+D}^{1/2}
            \exp\br{\frac{\pi i}{4}\br{{2{\vec{\de}}^t B^tC{\vec{\ep}}-{\vec{\de}}^t
            B^tD{\vec{\de}}-{\vec{\ep}}^t A^tC{\vec{\ep}}}}}
            \theta\ch{{\vec{\de}}}{{\vec{\ep}}}\br{\tau}
        \end{align}
    where $\zeta_\gamma$ is the eighth root of unity $\zeta_\gamma^8=1$ depending only on $\gamma$.
\end{theorem}

It is straightforward to see that by theorem \ref{thm:theta-modularity} 
Riemann theta constant $\theta_0^{16}$ is a weight-8 modular form w.r.t. $\Gamma(1,2)$.
More generally, Riemann theta constant $\theta_0^{8k}$ is a 
weight-$4k$ modular form w.r.t. $\Gamma(1,2)$ by same reasoning.

\begin{definition}[Even/odd characteristic]
    The characteristic $\ch {{\vec{\de}}}{{\vec{\ep}}}$ is called even if ${\vec{\de}} \cdot {\vec{\ep}}$ is even. 
    Otherwise, the characteristic is called odd.
\end{definition}

An interesting fact is that Riemann theta constants with odd characteristic are zero.
\begin{lemma}[\cite{mumford-theta}]
    $e \in \ZZ_2^g$ is odd. Then $\theta_e (\tau) \equiv 0$.
\end{lemma}

\begin{definition}[Original weight-8 Grushevsky functions]
    \begin{align}
        \xi_0^{8, (g)}[e] &= \theta^{16}_e \\
        \xi_1^{8, (g)}[e] &= \theta^{8}_e \sum_{e_1}^{N_e} \theta^8_{e+e_1} \\
        \xi_2^{8, (g)}[e] &= \theta^{4}_e \sum_{e_1}^{N_e} \theta^4_{e+e_1}\theta^4_{e+e_2}\theta^4_{e+e_1+e_2} \\
        ... \nonumber \\
        \xi_p^{8, (g)}[e] &= \sum_{e_1, ..., e_p}^{N_e} \Bigg \{
            \theta_e 
            \cdot \Big( \prod_i^p \theta_{e + e_i} \Big)
            \cdot \Big( \prod_{i < j}^p \theta_{e + e_i + e_j} \Big)
            % \cdot \Big( \prod_{i < j < k}^p \theta_{e + e_i + e_j + e_k} \Big)
            \cdot ...
            \cdot \theta_{e + e_1 + ... + e_p}
        \Bigg \} ^ {2^{4 - p}}
    \end{align}
    where $p=0..4$.
\end{definition}
Grushevsky ansatz in genus $g$ for superstring measures is defined as the linear combinations 
    of these weight-8 Grushevsky functions.
    We will not provide the coefficients here, because they are not important for this work.

By summing up the $\theta_e$ for different characteristics $e$ 
we obtain modular form w.r.t. somewhat large subgroup of $\Gamma^{(g)}$ 
(more precisely, w.r.t. subgroup $\Gamma_e$).

\begin{theorem}[\cite{morozov}]\label{thm:xi-8-are-modular}
    $\xi_{p}^{8, (g)}[e]$
    are weight-8 modular forms w.r.t. subgroup $\Gamma_e$.
\end{theorem}

Grushevsky functions are constructed as polynomials of Riemann theta constants, 
thus keeping the total polynomial degree divisible by 8 kills $\zeta_\gamma$.
This is why we want to increase the polynomial degree to 24 after 18: 
to keep the modularity property.
Another view on this move will be covered in section \ref{sec:lattice-theta-series} when we discover
the similar modularity property of lattice theta series.

Now, we can introduce the following definition:

\begin{definition}[Weight-12 Grushevsky functions]
    \begin{align}
        \xi_0^{12, (g)}[e] &= \theta^{24}_e \\
        \xi_1^{12, (g)}[e] &= \theta^{12}_e \sum_{e_1}^{N_e} \theta^{12}_{e+e_1} \\
        ... \nonumber \\
        \xi_p^{12, (g)}[e] &= \sum_{e_1, ..., e_p}^{N_e} \Bigg \{
            \theta_e 
            \cdot \Big( \prod_i^p \theta_{e + e_i} \Big)
            \cdot \Big( \prod_{i < j}^p \theta_{e + e_i + e_j} \Big) \cdot ...
            \cdot \theta_{e + e_1 + ... + e_p}
        \Bigg \} ^ {3 \cdot 2^{3 - p}}
    \end{align}
    where $p=0..3$.
\end{definition}

Grushevsky did not define or study weight-12 Grushevsky functions 
(because there is probably no 
application of them in superstring theory). They are original to the present work.
We have selected this naming to reflect that the construction originates from Grushevsky ansatz for
superstring measures.
We consider those Grushevsky functions purely as mathematical objects
and keep this naming throughout the article.

\begin{theorem}
    $\xi_{p}^{12, (g)}$ are weight-12 modular forms w.r.t. subgroup $\Gamma_e$.
\end{theorem}
\begin{proof}
    The proof is analogous to the proof of referenced theorem \ref{thm:xi-8-are-modular}.
\end{proof}

\section{Lattice theta series}\label{sec:lattice-theta-series}
In this section we will introduce the definition of lattice and the associated lattice theta series.
Also we will talk about the gluing theory, which allows classification
of lattices with fixed dimension.
\begin{definition}[Lattice]
    The set generated by all the integer linear combinations of a basis 
    $v_1, .., v_n \in \RR^n$ is called the lattice (generated by $v_1, .., v_n$).
    The number $n$ is called the dimension of the lattice.
\end{definition}

\begin{definition}[Dual lattice]
    Dual lattice $\Lambda^*$ for lattice $\Lambda \subset \RR^h$ 
    is the set of all vectors $u \in \RR^h$ s.t. 
    $\forall v \in \Lambda \quad u \cdot v \in \ZZ$.
\end{definition}

\begin{definition}[Unimodular lattice]
    Lattice $\Lambda$ is called self-dual or unimodular if it coincides with it's dual $\Lambda^*$. 
    An equivalent definition is that the Gram matrix $MM^t$ (where $M$ is the matrix formed with 
    the basis of $\Lambda$ as columns) has determinant 1.
\end{definition}


\begin{definition}[Lattice theta series]
    The genus-$g$ lattice theta series associated with the lattice $\Lambda \subset \RR^h$ is 
    a function on the Siegel space $\mathcal{H}^{(g)}$ defined as
    \begin{align}
        \vartheta^{(g)}_{\Lambda}(\tau) = 
        \sum_{(\vec p_1, ..., \vec p_g) \in \Lambda^g}
            \exp \Big(
                \pi i \sum_{k, l}({\vec{p_k}} \cdot {\vec{p_l}}) \tau_{kl}
            \Big)
    \end{align}
\end{definition}

\begin{lemma}[\cite{opsmy}]
    If 8 divides the number of dimensions $m$ of a unimodular lattice
    $\Lambda$, then the corresponding theta series $\vartheta^{(g)}_{\Lambda}$ for this lattice is a modular form of weight m/2 relative to $\Gamma^{(g)}(1,2)$.
\end{lemma}

This lemma shows us that if we want to match against the generalized Grushevsky construction,
which will consist of modular forms similarly to the original one,
then we should consider the lattice theta series of $m$-dimensional lattices where $m=8k$.

So the next step after the original $m=16$ is the $m=24$. And this gives another idea
on why the generalized Grushevksy construction should contain weight-12 modular forms.

\begin{theorem}[\cite{conway}, Gluing theory]\label{thm:gluing-theory}
    Suppose $L \subset \RR^n$ is an integer lattice containing a sublattice
    \begin{align}
        L_1 \oplus L_2 \oplus .. \oplus L_k \\
        \sum_i^k d_i=n
    \end{align}
    where every $L_i$ is also an integer lattice and $d_i$ is the dimension of $L_i$.
    Then $L$ is generated by $L_1 \oplus L_2 \oplus .. \oplus L_k$
    and some vectors $y^j, j=1..J$ of form
    \begin{align}
        y^j=y^j_1 + y^j_2 + .. + y^j_k
    \end{align}
    where each $y^j_i \in \RR^n$ is from $L_i^* \subset \RR^n$.
    Vectors $y^j=y^j_1+ ..+ y^j_n$ are called \emph{gluing vectors}.
    
    We will denote $L=\langle L_1 \oplus .. \oplus L_k, y^1, .., y^J \rangle$.
\end{theorem}

\begin{definition}[$A_n$ lattice]
    \begin{align}
        A_n = \{
            (x_1, .., x_{n+1}) \in \ZZ^{n+1}:
            \sum_i x_i = 0
        \}
    \end{align}
\end{definition}

\begin{definition}[$D_n, D_n^+$ lattices]
    \begin{align}
        D_n = \{
            (x_1, .., x_{n}) \in \ZZ^{n}:
            \sum_i x_i \quad \text{is even}
        \}
    \end{align}
    There is also another important lattice, $D_n^+$, which is defined only for even $n$:
    \begin{align}
        D_n^+ = D_n \cup \br{D_n + \br{\frac{1}{2}^n}}
    \end{align}
    Here and throughout the article 
    we use the notation $\br{a^n}$ to denote the vector $\br{a, a, .., a} \in \RR^n$.
\end{definition}

\begin{definition}[$E_7, E_8$ lattices]
    $E_8$ is another name for the lattice $D_8^+$. 
    And the lattice $E_7$ is defined as the sublattice of $E_8$ orthogonal to the 
    fixed minimal vector $v \in E_8$:
    \begin{align}
        E_7 = \{
            x \in E_8: x \cdot v = 0
        \}
    \end{align}
    (minimal vector of a lattice is any vector with smallest non-zero norm).
\end{definition}

\begin{theorem}\cite{conway}
    There are 8 16-dimensional unimodular lattices
\begin{center}
    \begin{tabular}{|c|c|c|}
        \hline
        Lattice theta series & Lattice & Gluing vector \\
        \hline
        $\vartheta_0$ & $\ZZ^{16}$ & -\\
        \hline
        $\vartheta_1$ & $\ZZ^8 \oplus E_8$ & -\\
        \hline
        $\vartheta_2$ & $\br{\ZZ^4 \oplus D_{12}}^+$ & $\Big( 0^4, \frac{1}{2}^{12} \Big)$ \\
        \hline
        $\vartheta_3$ & $\br{\ZZ^2 \oplus E_7 \oplus E_7}^+$ & 
            $\Big( \frac{1}{4}^6, -\frac{3}{4}^2, \frac{1}{4}^6, -\frac{3}{4}^2 \Big)$ \\
        \hline
        $\vartheta_4$ & $\br{\ZZ \oplus A_{15}}^+$ & 
            $\Big( \frac{1}{4}^{12}, -\frac{3}{4}^4 \Big), 
            \Big( \frac{1}{2}^8, -\frac{1}{2}^8 \Big),
            \Big( \frac{3}{4}^4, -\frac{1}{4}^{12} \Big)$
            \\
        \hline
        $\vartheta_5$ & $D_8 \oplus D_8$ & 
            $\Big( \frac{1}{2}^8, 0^7, 1 \Big)$
            \\
        \hline
        $\vartheta_6$ & $E_8 \oplus E_8$ & - \\
        \hline
        $\vartheta_7$ & $D_{16}^+$ & $\Big( \frac{1}{2}^{16} \Big)$ \\
        \hline
    \end{tabular}
\end{center}
For every lattice $\Lambda$ in 2-nd column the $\Lambda^+$ means the glued lattice 
in the sense of theorem \ref{thm:gluing-theory}. 
\end{theorem}

    OPSMY ansatz for superstring measures is defined as the linear combinations 
    of these lattice theta series.


\section{Weight-12 Grushevsky functions vs 24-dimensional lattice theta series}\label{sec:main}
We start by recalling the known result for weight-8 modular forms. Then we consider
our own case of weight-12 modular forms and 
formulate and prove the main results of this work: theorems \ref{thm:p-2} and \ref{thm:classification-of-lambda}.
\begin{theorem}[\cite*{dunin-barkowski}]
    In every genus $g$ there is an explicit connection between 
    weight-8 Grushevsky functions and lattice theta series  of unimodular 16-dimensional lattices, namely:
    \begin{align}
        \forall g \in\mathbb{Z}_{\geq 0} \quad \vartheta^{(g)}_p = 2^{-gp}\xi_p^{8, (g)}, \quad p=0..4
    \end{align}
\end{theorem}
    
    After the discussed theorem for $m=16$, the next question is
    what lattices in dimension $m=24$ correspond to the polynomials of
    Riemann theta constants given by our generalization: weight-12 Grushevsky functions. 
    Let's recall it:
    \begin{align}
        \xi_0^{12, (g)}[e] &= \theta^{24}_e \nonumber \\
        \xi_1^{12, (g)}[e] &= \theta^{12}_e \sum_{e_1}^{N_e} \theta^{12}_{e+e_1} \nonumber \\
        ... \nonumber \\
        \xi_p^{12, (g)}[e] &= \sum_{e_1, ..., e_p}^{N_e} \Bigg \{
            \theta_e 
            \cdot \Big( \prod_i^p \theta_{e + e_i} \Big)
            \cdot \Big( \prod_{i < j}^p \theta_{e + e_i + e_j} \Big) \cdot ...
            \cdot \theta_{e + e_1 + ... + e_p}
        \Bigg \} ^ {3 \cdot 2^{3 - p}} \nonumber
    \end{align}

    Clearly $p=0..3$, so we have to deal with 4 polynomials.

    The case $p=0$ is a trivial one and is proved in \cite{conway}.
    \begin{theorem}[\cite{conway}, $p=0$]
        \begin{align}
            \forall g \in\mathbb{Z}_{\geq 0} \quad \xi^{12, (g)}_0=\vartheta^{(g)}_{\ZZ^{24}}
        \end{align}
    \end{theorem}

    The case $p=1$ is also known, but we provide the proof to demonstrate the approach
    we follow for $p=2$.

    \begin{theorem}[$p=1$]
        \begin{align}
            \forall g \in\mathbb{Z}_{\geq 0} \quad \xi_1^{12, (g)}[0] = 2^{-g} \cdot \vartheta^{(g)}_{\ZZ^{12} \oplus D_{12}^+}
        \end{align}
    \end{theorem}
    \begin{proof}
        \begin{align}
            \frac{\xi_1^{12, (g)}[0]}{\theta_0^{12}} = \sum_{e} \theta_e^{12}
            &= \sum_{e} \Big(\sum_{\vec{n} \in \ZZ^g}
                \exp \big[ \pi i 
                    (\vec{n} + \frac{\vec{\de}}{2})^t \tau (\vec{n} + \frac{{\vec{\de}}}{2}))
                    + (\vec{n} + \frac{\vec{\de}}{2})^t {\vec{\ep}}
                \big]
            \Big)^{12} \nonumber \\
            &= \sum_e \sum_{\substack{\vec{n}^a \in \ZZ^g \\ a=1..12}}
                \exp \big[
                    \pi i \Big(
                        (\sum_{a=1}^{12} \vec{n}^a + \frac{{\vec{\de}}}{2})^t \tau 
                        (\sum_{a=1}^{12} \vec{n}^a + \frac{{\vec{\de}}}{2})
                        + 
                        (\sum_{a=1}^{12} \vec{n}^a + \frac{{\vec{\de}}}{2})^t {\vec{\ep}}
                    \Big)
                \big] \nonumber \\
            &= \sum_e \sum_{\substack{\vec{n}^a \in \ZZ^g \\ a=1..12}}
                \exp \big[
                    \pi i \Big(
                        (\sum_{a=1}^{12} \vec{n}^a + \frac{{\vec{\de}}}{2})^t \tau 
                        (\sum_{a=1}^{12} \vec{n}^a + \frac{{\vec{\de}}}{2})
                        + 
                        (\sum_{a=1}^{12} \vec{n}^a)^t {\vec{\ep}}
                    \Big)
                \big]
        \end{align}
        If any component of the vector $\sum_{a} \vec n^a$ is odd, then 
        there exist $2^{g-1}$ vectors ${\vec{\ep}} \in \ZZ^g$ where
        this component is equal to 1 and the same number of vectors where 
        this component is equal to 0.
        Every such pair of vectors ${\vec{\ep}}$ creates different in sign, 
        but equal in absolute value terms $\big(\exp(\pi i(..))\big)$ which 
        cancel out each other. So, we consider only vectors $\vec n^a$ s.t.
        the components of the vector $\sum_a \vec n^a$ are even. 
        The term $(\sum_a \vec n^a)^t {\vec{\ep}}$ is therefore always even and thus the 
        $\exp[..]$ does not depend on it since $2\pi i$-periodicity of $\exp$. So the 
        sum over all ${\vec{\ep}}$ (we had the sum over $e=\ch{{\vec{\de}}}{{\vec{\ep}}}$)
        is replaced by the multiplier $2^g$.

        After dividing by $2^g$ we get the following series:
        \begin{align}\label{al:alt-lambda-2}
            \sum_{{\vec{\de}} \in \ZZ_2^g} \sum_{\substack{\vec n^a \in \ZZ^g \\ a=1..12 \\ \sum_{a=1}^{12} \vec n^a \vdots 2}}
                \exp \Big(
                    \pi i \sum_{a=1}^{12}
                        (\vec n^a + \frac{{\vec{\de}}}{2})^t \tau
                        (\vec n^a + \frac{{\vec{\de}}}{2})
                \Big)
        \end{align}
        And it is easy to see now that the lattice theta series associated with lattice
        \begin{align}
            \Lambda_1 \cup \Big(\Lambda_1 + \big(\frac{1}{2}^{12}\big)\Big)
        \end{align}
        where 
        \begin{align}
            \Lambda_1 = \{ 
                (n^1, n^2, .., n^{12}) \in \ZZ^{12} |
                \sum_a n^a \vdots 2
            \}
        \end{align}
        coincides with the one given in \eqref{al:alt-lambda-2}.
        By definition the lattice $\Lambda_1 \cup \big(\Lambda_1+\big(\frac{1}{2}^{12}\big)\big)$
        is exactly the lattice $D_{12}^+$.
    \end{proof}

    The following result for $p=2$ is the main result of this work
    (alongside with theorem \ref{thm:classification-of-lambda}).
    \begin{theorem}[$p=2$]\label{thm:p-2}
        \begin{align}
            \forall g \in\mathbb{Z}_{\geq 0} \quad \xi_2^{12, (g)}[0] &= 2^{-2g} \cdot \vartheta^{(g)}_{\ZZ^6 \oplus \Lambda^+}
        \end{align}
        $\Lambda^+$ is the lattice obtained from $D_6^3=D_6\oplus D_6 \oplus D_6$
        by gluing it with
        \begin{align}
            \vec{\alpha} &= \Big(\frac{1}{2}^6, 0^6, \frac{1}{2}^6 \Big) \\
            \vec{\beta} &= \Big(0^6, \frac{1}{2}^6, \frac{1}{2}^6 \Big) \\
            {\vec{\de}} &= ( 1, 0^5, 1, 0^5, 1, 0^5 )
        \end{align}
    \end{theorem}
    Proof of the theorem will be separated into 4 lemmas.
    \begin{lemma}
        Lattice corresponding to the theta series
        \begin{align}
            2^{2g} \cdot \frac{\xi_2^{12, (g)}[0]}{\theta_0^6}
        \end{align}
        is
        the lattice $\Lambda^+$:
        \begin{align}
            \Lambda^+ = 
            \Lambda \cup (\Lambda + \vec{\alpha}) \cup (\Lambda + \vec{\beta}) \cup (\Lambda + \vec{\gamma})
        \end{align}
        where 
        \begin{align}
            \vec{\alpha}=\big(\frac{1}{2}^6, 0^6, \frac{1}{2}^6 \big) \\
            \vec{\beta}=\big(0^6, \frac{1}{2}^6, \frac{1}{2}^6\big) \\
            \vec{\gamma}=\big(\frac{1}{2}^6, \frac{1}{2}^6, 0^6\big)
        \end{align}.
    \end{lemma}
    \begin{proof}
        \begin{align}
            \frac{\xi_2^{12, (g)}[0]}{\theta_0^6} = 
                \sum_{e_1, e_2} \theta^6_{e_1} \theta^6_{e_2} \theta^6_{e_1+e_2}
        \end{align}
        Now, let $e_i = \ch{{\vec{\de}}_i}{{\vec{\ep}}_i}, i=1,2$ and consider
        \begin{align}
            \theta^6_{e_1} 
            &= \Big( \sum_{\vec{n}_1 \in \ZZ^g} \exp \big[ \pi i \big( 
                (\vec{n}_1 + \frac{{\vec{\de}}_1}{2})^t \tau (\vec{n}_1 + \frac{{\vec{\de}}_1}{2}) + 
                (\vec{n}_1 + \frac{{\vec{\de}}_1}{2})^t {\vec{\ep}}_1
                \big)
            \big] \Big)^6 \\
            &= \sum_{\vec{n}_1^1 \in \ZZ^g} (..) \cdot 
                \sum_{\vec{n}_1^2 \in \ZZ^g} (..) \cdot .. \cdot 
                \sum_{\vec{n}_1^6 \in \ZZ^g} (..) \\
            &= \sum_{\vec{n}_1^1 \in \ZZ^g}
                \sum_{\vec{n}_1^2 \in \ZZ^g} ..
                \sum_{\vec{n}_1^6 \in \ZZ^g} 
                \exp \big[ \pi i \big( 
                \sum_{a=1}^6 (\vec{n}_1^a + \frac{{\vec{\de}}_1}{2})^t \tau (\vec{n}_1^a + \frac{{\vec{\de}}_1}{2}) + 
                \sum_{a=1}^6 (\vec{n}_1^a + \frac{{\vec{\de}}_1}{2})^t {\vec{\ep}}_1
                \big) \big]
        \end{align}
        The term $\sum_{a=1}^6(\vec{n}_1^a+\frac{{\vec{\de}}_1}{2})^t {\vec{\ep}}_1$ splits on 
        $(\sum_{a=1}^6 \vec{n}_1^a)^t {\vec{\ep}}_1$ and 
        $6 \cdot \frac{{\vec{\de}}_1^t}{2} {\vec{\ep}}_1=3 {\vec{\de}}_1^t {\vec{\ep}}_1$.
        Since the $2\pi i$-periodicity of $\exp$, the latter is equivalent to ${\vec{\de}}_1^t{\vec{\ep}}_1$.
        The same holds for $\theta_{e_2}$ and $\theta_{e_1+e_2}$.
        Now
        \begin{align}
            \sum_{e_1, e_2} \theta^6_{e_1} \theta^6_{e_2} \theta^6_{e_1+e_2}
            &= \sum_{e_1, e_2} 
            \sum_{\substack{\vec{n}_1^a \in \ZZ^g \\ a=1..6}}
            \sum_{\substack{\vec{n}_2^a \in \ZZ^g \\ a=1..6}}
            \sum_{\substack{\vec{n}_{12}^a \in \ZZ^g \\ a=1..6}}
            \exp \Big [ \pi i \Big (
                \sum_{a=1}^6 (\vec{n}_1^a+\frac{{\vec{\de}}_1}{2})^t \tau (\vec{n}_1^a+\frac{{\vec{\de}}_1}{2}) \nonumber \\
                &+ \sum_{a=1}^6 (\vec{n}_2^a+\frac{{\vec{\de}}_1}{2})^t \tau (\vec{n}_2^a+\frac{{\vec{\de}}_1}{2}) + 
                \sum_{a=1}^6 (\vec{n}_{12}^a+\frac{{\vec{\de}}_1 + {\vec{\de}}_2}{2})^t \tau (\vec{n}_{12}^a+\frac{{\vec{\de}}_1 + {\vec{\de}}_2}{2}) \nonumber \\
                &+ (\sum_{a=1}^6 \vec{n}_1^a)^t {\vec{\ep}}_1 + 
                (\sum_{a=1}^6 \vec{n}_2^a)^t {\vec{\ep}}_2 +
                (\sum_{a=1}^6 \vec{n}_{12}^a)^t ({\vec{\ep}}_1 + {\vec{\ep}}_2) \nonumber \\
                &+ {\vec{\de}}_1^t {\vec{\ep}}_1 + {\vec{\de}}_2^t {\vec{\ep}}_2 + 
                ({\vec{\de}}_1 + {\vec{\de}}_2)^t ({\vec{\ep}}_1 + {\vec{\ep}}_2)
            \Big) \Big]
        \end{align}
        If $e$ is odd characteristic, then the corresponding 
        Riemann theta constant $\theta_e$ is zero.
        So, we can assume that ${\vec{\de}}_1^t {\vec{\ep}}_1$, ${\vec{\de}}_2^t {\vec{\ep}}_2$, and
        $({\vec{\de}}_1 + {\vec{\de}}_2)^t ({\vec{\ep}}_1 + {\vec{\ep}}_2)$ are even (otherwise the whole term
        $\exp(..)$ is zero).
        Again, since the $2\pi i$-periodicity of $\exp$, we can neglect them:

        \begin{align}
            \sum_{e_1, e_2} \theta^6_{e_1} \theta^6_{e_2} \theta^6_{e_1+e_2}
            &= \sum_{e_1, e_2} 
            \sum_{\substack{\vec{n}_1^a \in \ZZ^g \\ a=1..6}}
            \sum_{\substack{\vec{n}_2^a \in \ZZ^g \\ a=1..6}}
            \sum_{\substack{\vec{n}_{12}^a \in \ZZ^g \\ a=1..6}}
            \exp \Big [ \pi i \Big (
                \sum_{a=1}^6 (\vec{n}_1^a+\frac{{\vec{\de}}_1}{2})^t \tau (\vec{n}_1^a+\frac{{\vec{\de}}_1}{2}) \nonumber \\
                &+ \sum_{a=1}^6 (\vec{n}_2^a+\frac{{\vec{\de}}_1}{2})^t \tau (\vec{n}_2^a+\frac{{\vec{\de}}_1}{2}) + 
                \sum_{a=1}^6 (\vec{n}_{12}^a+\frac{{\vec{\de}}_1 + {\vec{\de}}_2}{2})^t \tau (\vec{n}_{12}^a+\frac{{\vec{\de}}_1 + {\vec{\de}}_2}{2}) \nonumber \\
                &+ (\sum_{a=1}^6 \vec{n}_1^a + \vec{n}_{12}^a)^t {\vec{\ep}}_1 + 
                (\sum_{a=1}^6 \vec{n}_2^a + \vec{n}_{12}^a)^t {\vec{\ep}}_2
            \Big) \Big]
        \end{align}

        Consider the vector $\sum_{a=1}^6 \vec{n}_1^a + \vec{n}_{12}^a$. If some component of it is odd, 
        then there exist $2^{g-1}$ binary vectors ${\vec{\ep}}_1 \in \ZZ_2^g$ 
        where the corresponding component is $1$, and the same number of vectors where 
        the corresponding component is $0$. Those vectors produce 
        terms that perfectly cancel each other.

        So, every component of the vector $\sum_{a=1}^6 \vec{n}_1^a + \vec{n}_{12}^a$
        must be even to give the non-zero result. 
        And if so, by $2 \pi i$-periodicity of $\exp$, we can neglect 
        the $(\sum_{a=1}^6 \vec{n}_1^a + \vec{n}_{12}^a)^t {\vec{\ep}}_1$ inside the $\exp(..)$.
        The same applies for vector $\sum_{a=1}^6 \vec{n}_2^a + \vec{n}_{12}^a$.
        
        Thus, the outermost sum over all characteristics 
        $e_1=\ch{{\vec{\de}}_1}{{\vec{\ep}}_1}, e_2=\ch{{\vec{\de}}_2}{{\vec{\ep}}_2}$ 
        is reduced to the sum
        over all binary vectors ${\vec{\de}}_1, {\vec{\de}}_2 \in \ZZ_2^g$ with coefficient $2^{2g}$
        and we get:
        
         \begin{align}
            \sum_{e_1, e_2} \theta^6_{e_1} \theta^6_{e_2} \theta^6_{e_1+e_2}
            &= 2^{2g}\sum_{{\vec{\de}}_1, {\vec{\de}}_2 \in \ZZ_2^g} 
            \sum_{\substack{
                \vec{n}_1^a \in \ZZ^g \\ a=1..6}}
            \sum_{\substack{\vec{n}_2^a \in \ZZ^g \\ a=1..6}}
            \sum_{\substack{\vec{n}_{12}^a \in \ZZ^g \\ a=1..6 \\
                \forall k \quad (\sum_{a=1}^6 \vec{n}_1^a + \vec{n}_{12}^a)_k \vdots 2 \\ 
                \forall k \quad (\sum_{a=1}^6 \vec{n}_2^a + \vec{n}_{12}^a)_k \vdots 2
            }}
            \exp \Big [ \pi i \Big (
                \sum_{a=1}^6 (\vec{n}_1^a+\frac{{\vec{\de}}_1}{2})^t \tau (\vec{n}_1^a+\frac{{\vec{\de}}_1}{2}) \nonumber \\
                &+ \sum_{a=1}^6 (\vec{n}_2^a+\frac{{\vec{\de}}_1}{2})^t \tau (\vec{n}_2^a+\frac{{\vec{\de}}_1}{2}) + 
                \sum_{a=1}^6 (\vec{n}_{12}^a+\frac{{\vec{\de}}_1 + {\vec{\de}}_2}{2})^t \tau (\vec{n}_{12}^a+\frac{{\vec{\de}}_1 + {\vec{\de}}_2}{2})
            \Big) \Big]
        \end{align}

        Now consider the lattice
        \begin{align}
            \Lambda = \{ 
                (n_1^1, .., n_1^6, n_2^1, .., n_2^6, n_{12}^1, .., n_{12}^6) \in \ZZ^{12}
                \big | \sum_{a=1}^6 n_1^a + n_{12}^a, \sum_{a=1}^6 n_2^a + n_{12}^a \quad \text{are even}
            \}
        \end{align}
        (here every $n^i_j$ represents one of the $g$ components of the vector $\vec n^i_j \in \ZZ^g$).

        The corresponding lattice theta series is:
        \begin{align}
            \vartheta^{(g)}_{\Lambda}(\tau) = \sum_{(\vec p_1, .., \vec p_g) \in \Lambda^g} 
                \exp(\pi i \sum_{k,l} ({\vec{p_k}} \cdot {\vec{p_l}}) \tau_{kl})
        \end{align}
        
        Let's consider the sum inside the $\exp(..)$:
        \begin{align}
            \sum_{k,l} ({\vec{p_k}} \cdot {\vec{p_l}}) \tau_{kl} 
            &= \sum_{k, l} \tau_{kl} \Big ( 
                \sum_{a=1}^6 n_1^a(k) \cdot n_1^a(l) +
                \sum_{a=1}^6 n_2^a(k) \cdot n_2^a(l) +
                \sum_{a=1}^6 n_{12}^a(k) \cdot n_{12}^a(l)
            \Big) \nonumber \\
            &= \sum_{a=1}^6 \sum_{k,l} n_1^a(k) \tau_{kl} n_1^a(l) + 
            \sum_{a=1}^6 \sum_{k,l} n_2^a(k) \tau_{kl} n_2^a(l) + 
            \sum_{a=1}^6 \sum_{k,l} n_{12}^a(k) \tau_{kl} n_{12}^a(l) \nonumber \\
            &= \sum_{a=1}^6 (\vec n_1^a)^t \tau \vec n_1^a + (\vec n_2^a)^t \tau \vec n_2^a + (\vec n_{12}^a)^t \tau \vec n_{12}^a
        \end{align}

        So the $\vartheta^{(g)}_\Lambda$ is almost what we want:
        \begin{align}
            \vartheta^{(g)}_\Lambda(\tau) = 
            \sum_{\substack{
                \vec{n}_1^a, \vec{n}_2^a, \vec{n}_{12}^a \in \ZZ^g \\
                a = 1..6 \\
                \forall k \quad (\sum_{a=1}^6 \vec{n}_1^a + \vec{n}_{12}^a)_k \vdots 2 \\ 
                \forall k \quad (\sum_{a=1}^6 \vec{n}_2^a + \vec{n}_{12}^a)_k \vdots 2
            }} \exp \Big [ \pi i \Big(
                \sum_{a=1}^6 (\vec{n}_1^a)^t \tau \vec{n}_1^a + 
                \sum_{a=1}^6 (\vec{n}_2^a)^t \tau \vec{n}_2^a + 
                \sum_{a=1}^6 (\vec{n}_{12}^a)^t \tau \vec{n}_{12}^a
            \Big) \Big]
        \end{align}

        Now, by gluing $\Lambda$ with vectors $\vec{\alpha}, \vec{\beta}, \vec{\gamma}$ (defined below) we obtain
        the lattice $\Lambda^+$:
        \begin{align}
            \Lambda^+ = 
            \langle \Lambda , \vec \alpha, \vec \beta, \vec \gamma \rangle
        \end{align}
        where 
        \begin{align}
            \vec{\alpha}=\big(\frac{1}{2}^6, 0^6, \frac{1}{2}^6 \big) \\
            \vec{\beta}=\big(0^6, \frac{1}{2}^6, \frac{1}{2}^6\big) \\
            \vec{\gamma}=\big(\frac{1}{2}^6, \frac{1}{2}^6, 0^6\big)
        \end{align}.

        Let's calculate the theta series for $\Lambda^+$. 
        If 
        ${\vec{p_k}} \in (\Lambda + \vec{\alpha})$ and ${\vec{p_l}} \in \Lambda$, then
        \begin{align}
            ({\vec{p_k}} \cdot {\vec{p_l}}) &= 
            \sum_{a=1}^6 (n_1^a(k) + \frac{1}{2}) \cdot (n_1^a(l) + \frac{1}{2}) \nonumber \\
            &+\sum_{a=1}^6 n_2^a(k) \cdot n_2^a(l) \nonumber \\
            &+\sum_{a=1}^6 (n_{12}^a(k) + \frac{1}{2}) \cdot (n_{12}^a(l) + \frac{1}{2})
        \end{align}
        If ${\vec{p_k}} \in (\Lambda + \vec{\alpha})$ and ${\vec{p_l}} \in (\Lambda + \vec{\beta})$, then
        \begin{align}
            ({\vec{p_k}} \cdot {\vec{p_l}}) &=
            \sum_{a=1}^6 (n_1^a(k) + \frac{1}{2}) \cdot (n_1^a(l) + \frac{1}{2}) \nonumber \\
            &+ \sum_{a=1}^6 (n_2^a(k) + \frac{1}{2}) \cdot (n_2^a(l) + \frac{1}{2}) \nonumber \\
            &+ \sum_{a=1}^6 (n_{12}^a(k) + \frac{1}{2} + \frac{1}{2}) 
                \cdot (n_{12}^a(l) + \frac{1}{2} + \frac{1}{2})
        \end{align}
        And the same for all other cases. The point is, if we fix the vectors
        $(\vec p_1, .., \vec p_g) \in (\Lambda^+)^g$ and construct the matrix where they represent rows, then
        we can consider
        $\vec{n}_1^1, .., \vec{n}_1^6, \vec{n}_2^1, .., \vec{n}_2^6, \vec{n}_{12}^1, .., \vec{n}_{12}^6$ as columns of this matrix
        (see the definition of $\Lambda$).
        For example, when ${\vec{p_k}} \in (\Lambda + \vec{\alpha})$, we get the $k$-th component of each vector
        $\vec{n}_1^1, .., \vec{n}_1^6, \vec{n}_{12}^1, .., \vec{n}_{12}^6$ shifted by $1/2$ 
        (at the same time $k$-th component of $\vec{n}_2^1, .., \vec{n}_2^6$ is not shifted).
        
        Now, we can view the $\sum_{k, l} ({\vec{p_k}} \cdot {\vec{p_l}}) \tau_{kl}$,
        as the sum "over rows" of our matrix. 
        The corresponding sum "over columns" could be written as:
        \begin{align}
            \sum_{{\vec{\de}}_1, {\vec{\de}}_2} \Big(
                \sum_{a=1}^6 (\vec{n}_1^a + \frac{{\vec{\de}}_1}{2})^t \tau (\vec{n}_1^a + \frac{{\vec{\de}}_1}{2})
                &+ \sum_{a=1}^6 (\vec{n}_2^a + \frac{{\vec{\de}}_2}{2})^t \tau (\vec{n}_2^a + \frac{{\vec{\de}}_2}{2}) \nonumber \\
                &+ \sum_{a=1}^6 (\vec{n}_1^a + \frac{{\vec{\de}}_1 + {\vec{\de}}_2}{2})^t \tau 
                    (\vec{n}_1^a + \frac{{\vec{\de}}_1 + {\vec{\de}}_2}{2})
            \Big)
        \end{align}
        where if we denote $k$-th row of our matrix by ${\vec{p_k}}$, then
        $({\vec{\de}}_1)_k=0, ({\vec{\de}}_2)_k=0$ means ${\vec{p_k}} \in \Lambda$;
        $({\vec{\de}}_1)_k=1, ({\vec{\de}}_2)_k=0$ means ${\vec{p_k}} \in (\Lambda + \vec{\alpha})$;
        $({\vec{\de}}_1)_k=0, ({\vec{\de}}_2)_k=1$ means ${\vec{p_k}} \in (\Lambda + \vec{\beta})$; 
        $({\vec{\de}}_1)_k=1, ({\vec{\de}}_2)_k=1$ means ${\vec{p_k}} \in (\Lambda + \vec{\gamma})$. \Big($({\vec{\de}}_1 + {\vec{\de}}_2)$ is the sum modulo 2\Big).
    \end{proof}

    \begin{lemma}
        The lattice $\Lambda^+=\langle \Lambda, \alpha, \beta, \gamma \rangle$ is the lattice
        \begin{align}
        \Lambda^+ = \Lambda \cup (\Lambda + \vec{\alpha}) \cup (\Lambda + \vec{\beta}) \cup (\Lambda + \vec{\gamma})
    \end{align}
    where 
    \begin{align}
        \Lambda &= \{ 
            (n_1^1, .., n_1^6, n_2^1, .., n_2^6, n_{12}^1, .., n_{12}^6) \in \ZZ^{12}
            \big | \sum_{a=1}^6 n_1^a + n_{12}^a, \sum_{a=1}^6 n_2^a + n_{12}^a \quad \text{are even}
        \} \\
        \vec{\alpha}&=\big(\frac{1}{2}^6, 0^6, \frac{1}{2}^6 \big) \\
        \vec{\beta}&=\big(0^6, \frac{1}{2}^6, \frac{1}{2}^6\big) \\
        \vec{\gamma}&=\big(\frac{1}{2}^6, \frac{1}{2}^6, 0^6\big) \\\
    \end{align}
    \begin{proof}
        Every vector in $\langle \Lambda, \vec \alpha, \vec \beta, \vec \de \rangle$
        can be expressed as 
        \begin{align}
            x \vec v + a \vec\alpha + b \vec\beta + c \vec\gamma
        \end{align}
        where $x, a, b, c \in \ZZ$. If all or neither of the $a, b, c$ are odd, 
        then the resulting vector
        lies in $\alpha$. If $a, b$ are odd and $c$ is even, then the resulting vector
        lies in $\Lambda + \gamma$. The cases where $b, c$ are odd, $a$ is even and $a, c$ are odd, 
        $b$ is even are analogous.
        If $a$ is odd, and $b, c$ are even, then the resulting vector lies in $\Lambda + \alpha$. 
        The same for case when only $b$ or $c$ is odd.
    \end{proof}
    
    \end{lemma}

    \begin{lemma}
    The lattice $\Lambda^+$ is equal to the lattice obtained by
    gluing $D_6^3=D_6 \oplus D_6 \oplus D_6$ with vectors $\vec{\alpha}$, $\vec{\beta}$, ${\vec{\de}}$, where 
    \begin{align}
        {\vec{\de}}&=\big(1,0^5, 1,0^5,1,0^5\big)
    \end{align}
    \end{lemma}

    \begin{proof}
        The lattice $\Lambda$ could be viewed as the sublattice of $\ZZ^{18}$ containing all
        the vectors of type $(e, e, e) \in \ZZ^{18}$ 
        and $(o, o, o) \in \ZZ^{18}$ where each $e \in \ZZ^6$ denotes the
        vector with even sum of components and each $o \in \ZZ^6$ denotes the vector with odd sum
        of components.

        The sublattice of $\Lambda$ which contains $(e, e, e)$ is precisely
        $D_6^3$. Let's show that 
        $D_6^3$ glued with $\vec{\alpha}, \vec{\beta}, {\vec{\de}}$ is 
        exactly $\Lambda^+$.

        Firstly, consider the lattice $D_6^3$ glued with ${\vec{\de}}$. It is easy to see
        that we obtain the lattice $\Lambda$: every vector of type $(e, e, e)$ is 
        by definition in $D_6^3$, and every vector of type $(o, o, o)$ could be obtained from
        the suitable vector of type $(e, e, e)$ by adding ${\vec{\de}}$.

        Now, glue the result with $\vec{\alpha}, \vec{\beta}$. It only remains to show that $\vec{\gamma}$
        is also generated:
        $$
        \vec{\alpha} + \vec{\beta} - \vec{\gamma}=(0^6, 0^6, 1^6)
        $$
        and since the $(0^6, 0^6, 1^6)$ is inside the lattice $\Lambda$ (as vector of type $(e, e, e)$)
        then the vector $\vec{\gamma}$ is inside the lattice $\Lambda$ as well.
    \end{proof}

    \begin{lemma}
        $\Lambda^+$ is unimodular.
    \end{lemma}

    \begin{proof}
        Let's construct the generating matrix $M$ for the lattice $\Lambda$ and see that 
        $\det MM^t=1$.

        It is known \cite{conway} that vectors
        \begin{align} 
            (-1, -1, 0, 0, 0, 0)    \nonumber \\
            (1, -1, 0, 0, 0, 0)     \nonumber \\
            (0, 1, -1, 0, 0, 0)     \nonumber \\
            (0, 0, 1, -1, 0, 0)     \nonumber \\
            (0, 0, 0, 1, -1, 0)     \nonumber \\
            (0, 0, 0, 0, 1, -1)
        \end{align}
        form the generating basis of $D_6$.
        Consider the basis of $D_6^3$:
        \begin{align}
            e^1_1&=(-1, -1, 0, 0, 0, 0, 0^6, 0^6)    \nonumber \\
            e^1_2&=(1, -1, 0, 0, 0, 0, 0^6, 0^6)     \nonumber \\
            e^1_3&=(0, 1, -1, 0, 0, 0, 0^6, 0^6)     \nonumber \\
            e^1_4&=(0, 0, 1, -1, 0, 0, 0^6, 0^6)     \nonumber \\
            e^1_5&=(0, 0, 0, 1, -1, 0, 0^6, 0^6)     \nonumber \\
            e^1_6&=(0, 0, 0, 0, 1, -1, 0^6, 0^6)     \nonumber \\
            e^2_1&=(0^6, -1, -1, 0, 0, 0, 0, 0^6)    \nonumber \\
            e^2_2&=(0^6, 1, -1, 0, 0, 0, 0, 0^6)     \nonumber \\
            e^2_3&=(0^6, 0, 1, -1, 0, 0, 0, 0^6)     \nonumber \\
            e^2_4&=(0^6, 0, 0, 1, -1, 0, 0, 0^6)     \nonumber \\
            e^2_5&=(0^6, 0, 0, 0, 1, -1, 0, 0^6)     \nonumber \\
            e^2_6&=(0^6, 0, 0, 0, 0, 1, -1, 0^6)     \nonumber \\
            e^3_1&=(0^6, 0^6, -1, -1, 0, 0, 0, 0)    \nonumber \\
            e^3_2&=(0^6, 0^6, 1, -1, 0, 0, 0, 0)     \nonumber \\
            e^3_3&=(0^6, 0^6, 0, 1, -1, 0, 0, 0)     \nonumber \\
            e^3_4&=(0^6, 0^6, 0, 0, 1, -1, 0, 0)     \nonumber \\
            e^3_5&=(0^6, 0^6, 0, 0, 0, 1, -1, 0)     \nonumber \\
            e^3_6&=(0^6, 0^6, 0, 0, 0, 0, 1, -1)     \nonumber \\
        \end{align}
        Replace the vector $e^1_2$ with ${\vec{\de}}$, $e^2_5$ with $\vec{\beta}$ and $e^3_5$ with $\vec{\alpha}$.
        It is easy to see that we still generate $D_6^3$:
        \begin{align}
            e^1_2&=e^1_1 + 2{\vec{\de}} + e^2_1 - e^2_2 + e^3_1 - e^3_2 \\
            e^2_5&=-5e^1_1 - 4{\vec{\de}} - 4e^1_3 - 3e^1_4 - 2e^1_5 - e^1_6 - 2e^2_1 + 2e^2_2 - 5e^3_1 - 0 - 4e^3_3 - 3e^3_4 - 2e^3_5 - 2e^3_6 \\
            e^3_5&=5e^1_1 + 4e^1_2 + 4e^1_3 + 3e^1_4 + 2e^1_5 + 1e^1_6 - 1e^2_1 - 4e^2_2 - 4e^2_3 - 3e^2_4 - 2e^2_5 - 2e^2_6 + 2e^3_1 - 2e^3_2 + 2e^3_6
        \end{align}
        Thus, the matrix $M^t$ constructed with this basis as rows:
        \begin{align}
    M^t = 
    \begin{bmatrix}{}
        -1 & -1 & 0 & 0 & 0 & 0 & 0 & 0 & 0 & 0 & 0 & 0 & 0 & 0 & 0 & 0 & 0 & 0 \\
    1 & 0 & 0 & 0 & 0 & 0 & 1 & 0 & 0 & 0 & 0 & 0 & 1 & 0 & 0 & 0 & 0 & 0 \\
    0 & 1 & -1 & 0 & 0 & 0 & 0 & 0 & 0 & 0 & 0 & 0 & 0 & 0 & 0 & 0 & 0 & 0 \\
    0 & 0 & 1 & -1 & 0 & 0 & 0 & 0 & 0 & 0 & 0 & 0 & 0 & 0 & 0 & 0 & 0 & 0 \\
    0 & 0 & 0 & 1 & -1 & 0 & 0 & 0 & 0 & 0 & 0 & 0 & 0 & 0 & 0 & 0 & 0 & 0 \\
    0 & 0 & 0 & 0 & 1 & -1 & 0 & 0 & 0 & 0 & 0 & 0 & 0 & 0 & 0 & 0 & 0 & 0 \\
    0 & 0 & 0 & 0 & 0 & 0 & -1 & -1 & 0 & 0 & 0 & 0 & 0 & 0 & 0 & 0 & 0 & 0 \\
    0 & 0 & 0 & 0 & 0 & 0 & 1 & -1 & 0 & 0 & 0 & 0 & 0 & 0 & 0 & 0 & 0 & 0 \\
    0 & 0 & 0 & 0 & 0 & 0 & 0 & 1 & -1 & 0 & 0 & 0 & 0 & 0 & 0 & 0 & 0 & 0 \\
    0 & 0 & 0 & 0 & 0 & 0 & 0 & 0 & 1 & -1 & 0 & 0 & 0 & 0 & 0 & 0 & 0 & 0 \\
    0 & 0 & 0 & 0 & 0 & 0 & 0 & 0 & 0 & 1 & -1 & 0 & 0 & 0 & 0 & 0 & 0 & 0 \\
    0 & 0 & 0 & 0 & 0 & 0 & \frac{1}{2} & \frac{1}{2} & \frac{1}{2} & \frac{1}{2} & \frac{1}{2} & \frac{1}{2} & \frac{1}{2} & \frac{1}{2} & \frac{1}{2} & \frac{1}{2} & \frac{1}{2} & \frac{1}{2} \\
    0 & 0 & 0 & 0 & 0 & 0 & 0 & 0 & 0 & 0 & 0 & 0 & -1 & -1 & 0 & 0 & 0 & 0 \\
    0 & 0 & 0 & 0 & 0 & 0 & 0 & 0 & 0 & 0 & 0 & 0 & 1 & -1 & 0 & 0 & 0 & 0 \\
    0 & 0 & 0 & 0 & 0 & 0 & 0 & 0 & 0 & 0 & 0 & 0 & 0 & 1 & -1 & 0 & 0 & 0 \\
    0 & 0 & 0 & 0 & 0 & 0 & 0 & 0 & 0 & 0 & 0 & 0 & 0 & 0 & 1 & -1 & 0 & 0 \\
    0 & 0 & 0 & 0 & 0 & 0 & 0 & 0 & 0 & 0 & 0 & 0 & 0 & 0 & 0 & 1 & -1 & 0 \\
    \frac{1}{2} & \frac{1}{2} & \frac{1}{2} & \frac{1}{2} & \frac{1}{2} & \frac{1}{2} & 0 & 0 & 0 & 0 & 0 & 0 & \frac{1}{2} & \frac{1}{2} & \frac{1}{2} & \frac{1}{2} & \frac{1}{2} & \frac{1}{2}
    \end{bmatrix} 
    \end{align}

    is the generating matrix for $\Lambda^+$. One can verify that $\det MM^t=1$.
    
    \end{proof}

    \begin{theorem}[\cite{conway}, Classification of unimodular lattices]
        All the unimodular lattices in dimension $m < 24$ which does not contain vectors 
        of unit norm are listed in \cite[Section~16.4, Table~16.7]{conway}.
    \end{theorem}

    \begin{theorem}\label{thm:classification-of-lambda}
        $\Lambda^+$ is the 3rd out of 4 18-dimensional unimodular lattices
        provided in classification 
        table in \cite[Section~16.4, Table~16.7]{conway}.

        Thus \begin{align}\label{eq:U183}
        \forall g \in\mathbb{Z}_{\geq 0} \quad  \xi^{12,(g)}_2 [0] = 2^{-2g}\, \vartheta^{(g)}_{Z^6\oplus U_{18,3}}
        \end{align}
        where  $U_{18,3}$ is the third 18-dimensional lattice in Conway's classification, and $\vartheta^{(g)}_{\ZZ^6\oplus U_{18,3}}$ is the respective theta series for the direct sum of lattices.
    \end{theorem}
    \begin{proof}
        It is easy to see that our lattice $\Lambda^+$ does not contain vectors of unit norm. All the 
        vectors of form $v + \vec{\alpha}, v + \vec{\beta}, v + \vec{\gamma}$ where $v \in \Lambda$ 
        has norm which is at least $3/2+3/2=3$. 
        Vectors of type $(e, e, e) \in \Lambda$ has norm at least 2,
        and vectors of type $(o, o, o) \in \Lambda$ has norm at least 3.

        And since the unimodularity of $\Lambda^+$ we can conclude that our lattice is presented
        in table in \cite[Section~16.4, Table~16.7]{conway}. 
        There are 4 lattices of dimension 18 there. 
        They are defined by sublattices: $A_{17}\oplus A_1$, $D_{10} \oplus E_7 \oplus A_1$,
        $D_6^3$, and $A_9^2$. Clearly, our lattice is the one containing the sublattice $D_6^3$.

        Formula \eqref{eq:U183} follows from theorem \ref{thm:p-2} and the above.
    \end{proof}
    
    

    The problem of finding the underlying lattice for the
    last case of $p=3$ (polynomial $\xi_3^{12, (g)}$)
    requires another approach because powers of $\theta$ are no longer even (see the proof of 
    theorem \ref{thm:p-2}).
    So, this question remains open.
\section{References}
\printbibliography

\end{document}